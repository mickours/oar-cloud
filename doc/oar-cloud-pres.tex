\documentclass{beamer}

\usepackage[T1]{fontenc}       
\usepackage[utf8]{inputenc}    % pour les accents (mettre latin1 pour windows au lieu de utf8)
\usepackage[frenchb]{babel}    % le documents est en français
\usepackage{amsmath}           % un packages mathématiques
\usepackage{xcolor}            % pour définir plus de couleurs 
\usepackage{graphicx}          % pour insérer des figures

\useoutertheme[height=0pt, width=80pt,left, hideothersubsections]{sidebar}
\usecolortheme{seahorse}
\setbeamercolor*{titlelike}{parent=structure}
\useinnertheme{circles}
\setbeamertemplate{frametitle}[default][right]
\setbeamertemplate{blocks}[rounded][shadow=true]


%Title info
\title[OAR cloud]{une infrastructure légère de Cloud Computing basé sur OAR}
\author{Mercier Michaël}
\date{}



% Faire apparaître un sommaire avant chaque section
\AtBeginSection[]{
   \begin{frame}
   \begin{center}{\Large Plan }\end{center}
   %%% affiche en début de chaque section, les noms de sections et
   %%% noms de sous-sections de la section en cours.
   \tableofcontents[currentsection, hideallsubsections]
   \end{frame} 
}
		

% Début de la présentation
% Contenu
\begin{document}
	% Page de titre
	\begin{frame}
		\titlepage
	\end{frame}
	
	
	% Sommaire
	\begin{frame} 
		\begin{center}{\Large Plan }\end{center}
		\tableofcontents[hidesubsections]
	\end{frame}
	
	
  \section{Projet OAR cloud}
		\begin{frame}
			\frametitle{OAR cloud}
			Les objectifs du projet
			\begin{itemize}
			  \item Faire une définition plus précise du sujet
			  \item Etât de l'art
			  \item Tester les technologies émergentes
			  \item Identification des problèmes
			  \item Conception de l'architecture générale
			\end{itemize}
			
% 				\begin{figure}
% 					\includegraphics[scale=0.15]{img/stm32l_discovery.jpg}
% 				\end{figure}
		\end{frame}
			
			
% 		\subsection{Test technologique}
% 			\begin{frame}
% 				\frametitle{Test technologique}
% 				
% 			\end{frame}
% 		
% 		\subsection{Identification des problèmes}
% 			\begin{frame}
% 				\frametitle{Identification des problèmes}
% 				
% 			\end{frame}
% 			
% 		\subsection{Conception générale}
% 			\begin{frame}
% 				\frametitle{Conception générale}
% 				
% 			\end{frame}
			
			
			
	\section{Le cloud computing}
	
		\subsection{Définitions}
			\begin{frame}
			  \frametitle{Définitions}
			  \begin{description}
			    \item[Cloud computing]
			    \item[IaaS] 
			  \end{description}
			\end{frame}
			
		\subsection{Outils}
			\begin{frame}
				\frametitle{Outils}
				
			\end{frame}
			
			
	\section{Virtualisation système}
	
		\subsection{Définitions}
			\begin{frame}
			  \frametitle{Définitions}
			\end{frame}
			
		\subsection{Outils}
			\begin{frame}
				\frametitle{Outils}
			\end{frame}
			
			
	\section{Virtualisation réseaux}
	
		\subsection{Définitions}
			\begin{frame}
			  \frametitle{Définitions}
			\end{frame}
			
		\subsection{Outils}
			\begin{frame}
				\frametitle{Outils}
			\end{frame}
			
			
	\section{Gestion de projet}
	
		\subsection{Journal}
			\begin{frame}
			  \frametitle{Journal}
			\end{frame}
			
		\subsection{Conception}
			\begin{frame}
				\frametitle{version 0.1}
			\end{frame}
			\begin{frame}
				\frametitle{version 0.3}
			\end{frame}
		
		\subsection{Organisation du travail}
			\begin{frame}
			  \frametitle{Organisation du travail}
			  \begin{itemize}
			    \item gestion d'équipe
			    \item contrainte du travail individuel
			  \end{itemize}
			\end{frame}
		
		
	\section{Bilans}
	  \begin{frame}
      \frametitle{Bilans}
		  \begin{itemize}
		    \item Bilan personnel
		    \item Bilan technique
		  \end{itemize}
	  \end{frame}
		
% 		\begin{alertblock}{}
% 		\end{alertblock}

% 		\begin{exampleblock}{ Points positifs }
% 		\end{exampleblock}
			
			
\end{document}

